\documentclass{article}
\renewcommand{\labelenumi}{(\alph{enumi})}
\usepackage[utf8]{inputenc}
\author{Greg Trent}
\title{Homework 1}
\date{08/24/2017}
\begin{document}
\maketitle
(Due 09/01/2017)

Properties:
\begin{enumerate}
\item (normality) $m(I)=$ the length of $I$ for every interval $I$;
\item (translation-invariance) $m(x+A)=m(A)$ for every $A$; and
\item (countable additivity) $m(\bigcup A_n)=\sum m(A_n)$ for every sequence of pairwise disjoint sets $A_n$.
\end{enumerate}

\section{Ex 1.1}
  Show that the properties (a)--(c) imply finite additivity: If $A$ and $B$ are disjoint then $m(A\cup B)=m(A)+m(B)$.

\subsection{Proof:}
By property c, countable additivity, we know that $m(\bigcup A_n)=\sum m(A_n)$ for every sequence of disjoint sets. "Every" is defined as "countable or finite" in this situation, which includes finite sequences, which implies finite additivity. QED. 

\section{Ex 1.2}
  Show that the properties (a)--(c) imply the inclusion--exclusion principle: For any sets $A,B$ we have $m(A\cup B)+m(A\cap B)=m(A)+m(B)$.

\subsection{Proof:}
If $A$ and $B$ are disjoint, then we know that $m(A \cup B) = m(A) + m(B)$, per countable additivity. This also means that $A \cap B$ is empty, so $m(A \cap B) = 0$. Then $m(A \cup B) + m(A \cap B) = m(A \cup B) + 0 = m(A) + m(B)$. So this checks out, by property c.  

If $A$ and $B$ are not disjoint, then there's overlap - some set $C$
\section{Ex 1.3}
  Complete the details of the proof that if there is a measure on $R$ satisfying properties (a)--(c), then there is a measure on $[0,1)$ (with addition modulo $1$) satisfying properties (a)--(c).

\section{Ex 2.1 (Tao 1.1.1)}
 Show that the class of elementary sets is closed under the operations: union, intersection, set difference, symmetric difference, and translation.
 
 \section{Ex 2.2}
Prove: The elementary measure function $m$ satisfies
  \begin{itemize}
  \item (monotonicity) $m(E)\leq m(F)$ for elementary sets $E\subset F$; and
  \item (finite subadditivity) $m(E\cup F)\leq m(E)+m(F)$ for elementary $E,F$.
\end{itemize}

\end{document}
