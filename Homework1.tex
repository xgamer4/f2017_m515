\documentclass{article}
\usepackage[utf8]{inputenc}
\author{Greg Trent}
\title{Homework 1}
\date{08/24/2017}
\begin{document}
\maketitle
(Due 09/01/2017)
\section{Tao ex. 1.1.1 (Boolean Closure)}
Show that if $E, F \in R^d$ are elementary sets, then the union of $E \cup F$, the intersection $E \cap F$, and the set theoretic difference $E \setminus F := \{ x \in E : x \notin F\}$, and the symmetric difference $E \triangle F := (E \setminus F) \cup (F \setminus E)$ are also elementary. 

If $x \in R^d$ show that the translate $E+x := \{y+x : y \in E\}$ is also an elementary set.

\begin{itemize}
\item $E \cup F$



\item $E \cap F$ 


\item $E \setminus F$


\item $E \triangle F$



\item The translate $E+x$

\end{itemize}

\section{Tao ex. 1.1.2}
Give an alternate proof of Lemma 1.1.2(ii) by showing that any two partitions of $E$ into boxes admit a mutual refinement into boxes that arise from taking Cartesian products of elements from finite collections of disjoint intervals. 

\subsection{Lemma 1.1.2(ii)}
Let $E, F \in R^d$ be an elementary set. If $E$ is partitioned as a finite union $B_1 \cup ... \cup B_k$ of disjoint boxes, then the quantity $M(E) := |B_1| + ... + |B_k|$ is independent of the partition. 


\section{Tao ex. 1.1.3}

\end{document}
